
\chapter{Systemu wspomagania sterowania}
Celem tego rozdzia�u jest opracowanie prostego systemu wspomagania sterowania (CAS - Control Augmentation System). 

Projekt systemu CAS:\\
* Stany samolotu podlegaj�ce regulacji - pr�dko�ci k�towe pochylania, przechylania i odchylania.\\
* Zestaw regulator�w jednowymiarowych dla ka�dego kana�u sterowania\\

Analiza uk�adu zamkni�tego (z w��czonym zar�wno CAs jak i systemem t�umienia flatteru):\\
* Por�wnanie z wynikami otrzymanymi dla skrzyd�a sztywno zamocowanego\\
* Symulacja manewr�w - CAS i system t�umienia flatteru mog� rywalizowa� o sygna� steruj�cy.\\

System b�dzie stanowi� punkt odniesienia dla kolejnego rozdzia�u.


\chapter{Kompleksowy projekt systemu wspomagania sterowania i t�umienia flatteru}

Pr�ba ca�o�ciowego rozwi�zania problemu - projekt jednego regulatora pe�ni�cego funkcje Systemu Wspomagania Sterowania i Systemu T�umienia flatteru.
